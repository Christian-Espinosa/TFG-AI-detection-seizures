\documentclass[10pt,a4paper,twocolumn,twoside]{article}
\usepackage[utf8]{inputenc}
\usepackage[catalan]{babel}
\usepackage{multicol}
\usepackage{graphicx}
\usepackage{fancyhdr}
\usepackage{times}
\usepackage{titlesec}
\usepackage{multirow}
\usepackage{lettrine}
\usepackage[top=2cm, bottom=1.5cm, left=2cm, right=2cm]{geometry}
\usepackage[figurename=Fig.,tablename=TAULA]{caption}

\usepackage{biblatex}
\addbibresource{ref.bib}

\captionsetup[table]{textfont=sc}

\author{\LARGE\sffamily Christian Espinosa Reboredo}
\title{\Huge{\sffamily Recognition of epileptic seizures from EEG data}}
\date{}

\newcommand\blfootnote[1]{%
  \begingroup
  \renewcommand\thefootnote{}\footnote{#1}%
  \addtocounter{footnote}{-1}%
  \endgroup
}

%
%\large\bfseries\sffamily
\titleformat{\section}
{\large\sffamily\scshape\bfseries}
{\textbf{\thesection}}{1em}{}

\begin{document}
\nocite{*}
\fancyhead[LO]{\scriptsize AUTOR: TÍTOL DEL TREBALL}
\fancyhead[RO]{\thepage}
\fancyhead[LE]{\thepage}
\fancyhead[RE]{\scriptsize EE/UAB TFG INFORMÀTICA: TÍTOL (ABREUJAT SI ÉS MOLT LLARG)}

\fancyfoot[CO,CE]{}

\fancypagestyle{primerapagina}
{
   \fancyhf{}
   \fancyhead[L]{\scriptsize TFG EN ENGINYERIA INFORMÀTICA, ESCOLA D'ENGINYERIA (EE), UNIVERSITAT AUTÒNOMA DE BARCELONA (UAB)}
   \fancyfoot[C]{\scriptsize ``Mes'' de 20xx, Escola d'Enginyeria (UAB)}
}

%\lhead{\thepage}
%\chead{}
%\rhead{\tiny EE/UAB TFG INFORMÀTICA: TÍTOL (ABREUJAT SI ÉS MOLT LLARG)}
%\lhead{ EE/UAB \thepage}
%\lfoot{}
%\cfoot{\tiny{February 2015, Escola d'Enginyeria (UAB)}}
%\rfoot{}
\renewcommand{\headrulewidth}{0pt}
\renewcommand{\footrulewidth}{0pt}
\pagestyle{fancy}

%\thispagestyle{myheadings}
\twocolumn[\begin{@twocolumnfalse}

%\vspace*{-1cm}{\scriptsize TFG EN ENGINYERIA INFORMÀTICA, ESCOLA D'ENGINYERIA (EE), UNIVERSITAT AUTÒNOMA DE BARCELONA (UAB)}

\maketitle

\thispagestyle{primerapagina}
%\twocolumn[\begin{@twocolumnfalse}
%\maketitle
%\begin{abstract}
\begin{center}
\parbox{0.915\textwidth}
{\sffamily
%\textbf{Resum--}

%\end{abstract}
%\bigskip
%\begin{abstract}
%\bigskip
%\\
\textbf{Abstract--}An electroencephalogram (EEG) is a test that detects electrical activity of the brain. Since 
1924 this procedure has been done countless times to obtain brain activity. This paper tries to go a step further 
to understand electroencephalography better using deep learning algorithms. The data used in this paper is a public 
dataset CHB-MIT\cite{goldberger2000physiobank} of recordings of paediatric subjects with intractable seizures. Different methods of data management 
are done and documented to make the most of the algorithms used. The objective is to train an algorithm to acknowledge 
when the subject is having a seizure.
\\
\\
\textbf{Keywords-- }electroencephalogram, deep learning, brain activity, classification, EEG analysis\\
}

\bigskip

{\vrule depth 0pt height 0.5pt width 4cm\hspace{7.5pt}%
\raisebox{-3.5pt}{\fontfamily{pzd}\fontencoding{U}\fontseries{m}\fontshape{n}\fontsize{11}{12}\selectfont\char70}%
\hspace{7.5pt}\vrule depth 0pt height 0.5pt width 4cm\relax}

\end{center}

\bigskip
%\end{abstract}
\end{@twocolumnfalse}]

\blfootnote{$\bullet$ E-mail de contacte: 1459024 @uab.cat}
\blfootnote{$\bullet$ Menció realitzada: Computació }
\blfootnote{$\bullet$ Treball tutoritzat per: Aura Hernández Sabaté (Ciencies de la Computació)}
\blfootnote{$\bullet$ Curs 2021/22}

\section{Introduction}
\leavevmode\\
\lettrine[lines=3]{A}{n} epileptic seizure is a period of symptoms due to abnormally excessive or synchronous neuronal activity in the brain. This can cause different effects like uncontrolled shaking movements involving much of the body, parts of the body or subtle momentary loss of awareness. In order to understand this issue, it is important to understand how neurons work and interact with each other to conserve what we call consciousness represented as brain activity and brainwaves. 
\\
Neural oscillations are rhythmic or repetitive patterns of neural activity in the central nervous system which can be driven by mechanisms within individual neurons or by interactions. Since 1824 neural oscillations have been observed, fifty years later intrinsic oscillatory behaviour was encountered in vertebrate neurons, but the purpose of these is yet to be fully understood.
\\
As the main objective is to classify seizures from brain activity first of all, it will be needed an inside view on how the brain works to have a hint on how and what is done to extract or intercept information from the neurons to process externally in a computer. This information is available annexed in this paper just to have an overview to further understand the subject, but this matter is not the main purpose of this paper as it specifies in data processing, architecture, model strategies and classification results.
\\
In this TFG we detail the complete process of epileptic seizure detection, from data processing to seizure recognition. Because this TFG works as a pipeline of different stages, an insight view of each is done, starting from data processing from a well-known database (CHB-MIT) of encephalograms collected from 23 subject with interactable seizures that has been used in previous research. Followed by the strategy of the architecture used to classify the signals to finally understand the classified results into seizure and not seizure.

\section{Related Work}
\label{sec-related-work}
A lot of research has been made of the brain to further understand it’s capabilities using deep learning algorithms. Below are different projects attempting to interpret and process EEG data in order to define a baseline of what has been done so far. 
\\
\subsection{Mental Workload Detection based on EEG Analysis}
\label{subsec-work1}
A study of mental workload is done in order to work more efficiently, healthier and to avoid accidents since workload compromises both performance and awareness. The use of EEG signals has a high correlation with specific cognitive and mental states such as workload, proposing a binary neural network to classify EEG features across different mental workloads.
\\
Mental workload is defined as “the cognitive and psychological effort to conclude a task”, observing that depending if workload is too heavy or too light this can affect human performance. Also, since workload involves neuro-physiologic, and perceptual processes it is affected by individual capabilities, motivation to preform, physical and emotional state which it’s multifaceted nature of workload prevents it of studying directly but it is feasible to infer this from a number of quantifiable variables.
\\
There are two main categories to measure workload:
\\
\begin{itemize}
\item Subjective measures: Being the most used to asses mental workload, the NASA Task Load Index (TLX) a prominent way to gain insight on perceived workload from the subject based on a weighted average of six sub-variables: mental demand, physical demand, temporal demand, performance, effort and frustration. Widely used in aviation to assess mental workload of the pilot but it is highly subjective.
\item Physiological measures: Providing a more reliable data by measuring physiological dynamic changes which cannot be controlled consciously, so that is why it’s more reliable. Readings such as electrocardiogram (ECG), electromyograph, electroencephalogram (EEG), photoplethysmography, respiration rate sensors, electro-dermal activity (EDA), oxygen density in the blood in the brain, and eye movement trackers... The combination of inputs reports better accuracy than the analysis if each one independently.
\end{itemize}
\leavevmode\\
The approach is to investigate the ability of 1D-CNN models to recognise two types of mental load from EEG signals and to generalise the model to a population not seen in the training set. They use N-back test (memory demanding games requiring the resolution of simple arithmetic operations adjusting workload) to induce low and medium workload and to classify a simple neural network (NN) it’s trained using only the power spectrum of theta waves. Proposing a personalized model for each individual and a generalist one, preform the models in a dataset of 16 subjects showing outstanding results in a leave-one-out subject test and so generalizing to new unseen subjects.
\\
The method to obtain data all 16 subjects had to first watch a 10-minute relaxing video and afterwards do the N-back-test low, medium and high difficulty. Finally subjects ask a TLX questionnaire for subjective perception of the test difficulty and workload.
\\
To get the data itself EEG recordings were done using EMOTIV EPOC+ headset which has 14 electrodes placed according to the 10/20 system which provide raw data and power spectrum for the main brain rhythms (theta, alpha, beta low, beta high, and), at 128 Hz and 8 Hz, respectively. Data had to be filtered from noise with Inter Quartile Range (IQR) strategy to detect outlier values associated to muscular movement wave peaks. In this work, power spectrum used was from theta wave (4-8Hz) sampled at 8Hz feed into the models in 5 second windows of every electrode (14 EEG sensors). Variables were normalized to have 0 mean and
sigma=1 using the mean and standard deviation of the training set. 
\\
Networks have a hidden layer of 128 neurons with ReLU as the activation function. Before the classification layer a dropout layer was added to avoid overfitting and all models have been trained using weighted cross-entropy loss, to compensate the different lengths of base line and workload phases which introduces unbalance in data samples.
\\
In conclusion its reported 95\% confidence interval for each class computed for all subjects and recall above 90\%. It is suggested to use longer windows to capture EEG non stationary nature.
\\
%\subsection{EEG SIGNAL DIMENSIONALITY REDUCTION AND CLASSIFICATION USING TENSOR DECOMPOSITION AND DEEP CONVOLUTIONAL NEURAL NETWORKS}
%\label{subsec-work2}

\section{Objectives}
\label{sec-objectives}
There are two main objectives in this TFG. The first objective is to find the best way to treat and analyse before feeding it to the deep learning algorithms. For example, as a continuous stream of sequence, dividing data per subject…  Or trying different test/train strategies such as leave one out, 50/50… The second main objective is to try to find which algorithms architectures have better results. There are different models offered by the research group IAM from the CVC.
\\
To fulfil the objectives is crucial to define smaller objectives to make work easier. Within the first objective there are several important parts. 
\\
\begin{itemize}
  \item Raw data must be readable, as the data base CHB-MIT\cite{goldberger2000physiobank} is in European Data Format (EDF), a standard file format designed for exchange and storage of medical time series, so all files in the dataset are “.edf”. A script has been programmed to save .edf files into .parquet format.
  \item Setting different functions to filter data making sure data is okey and fits certain constraints. 
  \item Get the labelling data, to have a ground truth from the recorded data. This part is essential to understand if the model works as expected.
  \item Define different functions to define how data enters the model to be trained. There are many ways information can be extracted from data. 
\end{itemize}
\leavevmode\\
The next objective after obtaining data is to use different models provided by the research group to IAM:
\\
\begin{itemize}
  \item Each model needs to be configured to accept the dimensionality of the data fed to it.
  \item Work different models to choose what models give better answers form input data.
  \item After all the models results, an overview is done to understand the results and conclude the best way to treat this database, for further investigation.
\end{itemize}
\leavevmode\\

\section{Methods}
\subsection{Dataset}
The dataset is data collected from the Children’s Hospital Boston, consisting in EEG recordings of subjects with intractable seizures. The folders classify in 23 cases from 22 subjects (case chb21 and chb1 are the same but 1.5 years apart). The subject’s personal information gender and age is in a separate file called SUBJECT-INFO added in this paper as subject\_info.csv.
\\\\
Each case contains between 9 and 42 edf files. There are edf files of EEG signals without seizures and others with recordings of seizures, these defined in RECORDS-WITH-SEIZURES. The files with seizures have the extension edf.seizures which disables the possibility of accessing the file with a normal edf reader library.
\\\\
Most cases have 1 hour of EEG recordings, but some have 1 to 4 hours depending on the case, split between 9 to 42 edf recordings, recorded at 256Hz in 16 bit resolution. It’s important to note, some subjects had hardware interruptions while the recording of the EEG, and so when there is an interruption, it’s noted in the summary.txt file. This kind of interruptions are a problem to get information normally, because the disposition of the electrodes change making it harder to control the disposition of the electrodes and the EEG might not work for a sequential approach, for example, if it’s important there is hypothetically an order on how a seizure comes to be, this file would certainly be discarded. To take into account this file the script to process the data in this TFG should be programmed to do so. For now, the objective is different, this project just classifies if there is or not a seizure, but for further development it should be considered. The data base is very large containing enough uninterrupted data to work with.
\\\\
The data used to train the model in this TFG is data from subject 1 to subject 10. Also only few .edf in each subject’s folders have any seizure. If every .edf was used, there would be a lot of data labelled as not seizure than seizure data, so from each subject only the files with seizure are used. Even so, handling only files with seizures the dataset is unbalanced, so for future development two strategies should be considered:

\begin{itemize}
  \item Files should be cut to have a balance of labelled data of 50\% data with seizure and 50\% data without. This strategy could end up un subsampling, considering there are few seizures in the hole dataset.
  \item When training the model, the criteria of cross Entropy should be weighted. To consider no seizure less important than seizure data.
\end{itemize}
\leavevmode\\
\subsection{Data processing}
Because in this TFG the CHB-MIT Scalp EEG Database is being used and all files are in format edf, a first script has been needed to process data, called “03\_ReadEDF.py”.
\\\\
In the “03\_ReadEDF.py” script there are different options on how raw data is imported, and also there are options on the way to execute them. During the development of this TFG many tests have been done, there for there are two different ways to execute the script:
\\
\begin{itemize}
  \item Single execution, where the subject number and the edf file of the subject needs to be provided to execute the script for this single file. 
  \item Multiple executions, where the number of subjects is provided. The script will go through all the first n subjects defined.
\end{itemize}
\leavevmode\\
It’s designed to extract the files from a specific folder hierarchy where all the encephalograms are classified by subjects. For simplicity this script obtains, filters, plots, and saves in parquets all input data. Afterword’s it also labels the data and splits it into windows so the model can process data easily. 
\\\\
The script will automatically label all raw data using the summary file in each subject’s folder, so it’s important for it to be present or a label execution error will pop up. The files edf.seizures in every subject’s folder were unreadable, even reading the binary was a failure. The script will make sure the file has all the data from the desired electrodes, this is important because there were hardware problems while recording the edfs, some files have gaps or lack some data, if any edf file has this problem it will automatically be excluded and the user will be notified. Each one has 22 different channels, which are the electrodes of the subject. In order to label the data, a new column is created (the 23rd) as seizure with the information of every row being a seizure or not and also a 24th column to set the observation windows for the model.
\\\\
Filtering data is done by first setting a maximum range from 0.5hz to 50hz, and afterwards only by changing the name of a parameter it can be changed to delta, theta, alpha, beta or gamma’s range frequencies in a dictionary added by default. Everything is modular so it can be changed any time with any range of frequencies. All data is saved in parquet format in a different folder, if plotting is enabled it will plot each subject’s data.
\\\\
Once the data is filtered and labelled it`s saved two numpys, one “file\_data\_x.npy” and “file\_data\_y.npy”. In data\_x the file has a numpy array of three dimensions containing number of windows, electrodes and values. In data\_y there are two dimensions, number of windows and window\_seizure which defines if the window has a seizure in it with a 1 or not with a 0. 
\\\\
With this, all the parquets and arrays are specifically saved and ordered to ensure easy access and fast comprehension of the hierarchy of folders. The database folder has every subject in a separate folder and inside individual folders for edf, parquets, numpys and results.
\\
\begin{figure}[h!]
    \caption{Filtered data in Theta range from electrode FP1-F7, subject 1 file 3}
    \centering
    \includegraphics[width=0.5\textwidth]{img/seizurenoseizure.png}
\end{figure}

\leavevmode\\
\subsection{Network}
An already done deep learning algorithm is used from the research group IAM from the CVC, which is working on a framework to determine the optimal architecture for cognitive state recognition from EEG signals, with the objective to answer different questions:
\\
\begin{itemize}
  \item How to combine the signals to create the input features for feature extraction? In this case, having 14 sensors x 5 wavelengths, so 70 raw signals. These signals can be concatenated, or projected.
  \item Which neural network is the best performer?
  \item Is it better to ensemble the different classifiers before combining the signals?
\end{itemize}
\leavevmode\\
This model was originally intended to study brain workload, so, with the help of this model it’s changed to fulfil the objective of clinic seizure detection. In this TFG, different strategies are applied on the input data of the algorithm to further study it’s capabilities as well as using different models to compare results between them.
\\\\
Once all desired raw data is filtered and saved, the second script to execute is “04\_MExecution.py”, this one is in charge of the execution of the model, training and testing to obtain the results classifying the data. All hyperparameters are defined at the beginning of the script as well as the declaration and initialization of the model. There are different typs of execution for development that can be enabled by boolean variables commented in the script. 
\\
\begin{itemize}
  \item CheckModel is the first way to execute the model, this one is needed to make sure the model works as intended before using real data to train it. It uses the selected model and feeds to it random data using torch with the defined parameters in the function.
  \item If the model works with no isues with CheckModel, then the training execution can begin. This first phase selects every numpy from n subjects defined at the beginning and trains the model with all the data. Every time the model finishes the epochs of a file, it saves the model in the database.
  \item In the second phase the previous trained model is loaded to test it and data is defined to set the normalization scalers for testing. For further development this should be changed to make an average of scalers by all trained data for example, or normalize data before the script’s execution. Once the model is tested a classification report is done using sklearn library and saved to a results folder in the subject’s folder.
\end{itemize}
\leavevmode\\
\\
The script uses numpy arrays as input data stored as previously mentioned in data\_x and data\_y. These files depending on the strategy of the script can be processed one last time to ensure good results from the classification. In this TFG data when loaded uses the strategy stated previously as phase 1 and 2, for training and testing. But the script has the option to split the files in train and test by a percentage variable. For the executions done this variable is set to 1 (100\%) on training, and 0 on testing phase.
\\\\
During the implementation of this script, a lot of problems of dependency on critical libraries has happened. Starting by cuda, for faster results it is used in all models, but it might not work if the architecture of the graphics card is too old. It is also not compatible with python 3.10 which was the version being worked on at the moment, it had to be switched to an environment with python 3.8 to avoid further issues. The script will be executed with cuda if the libraries are available, if not it will automatically work with the CPU.
\\\\
When it’s time to save the model, it’s saved in a folder called “trys” in the root directory, like if it was another subject but it only contains pt files, ordered in it by the date of execution.
\\\\
The models used in the TFG have similar structures. The first one CNN\_ConcatInput and the second one CNN\_ProjOut\_Conv. As the names suggest these models are Convolutional Neural Networks. Both models have 3 base layers of convolution defined by a function ConVNet which defines 3 different layers using torch, the first one being 16@22x19. The first parameter as number of input features defined by a constant variable defined in the constructor of the model, the second parameter defines the size of the data, 22 being the channels, in this case 
\\
\leavevmode\\
\begin{figure}[h!]
  \caption{Scheme of data dimension transformation through model CNN\_ConcatInput}
  \centering
  \includegraphics[width=0.5\textwidth]{img/FeatureProjectorModel CHM.png}
\end{figure}
\leavevmode\\
During the implementation of this procedure, a lot of problems of dependency on critical libraries has happened. Starting by cuda, for faster results it is used in all models, but it might not work if the architecture of the graphics card is too old. It is also not compatible with python 3.10 which was the version being worked on at the moment, it had to be switched to an environment with python 3.8 to avoid further issues.
\\


\leavevmode\\

\begin{figure}[h]
  \caption{Scheme of data processing }
  \centering
  \includegraphics[width=0.5\textwidth]{img/FeatureProjectorModel.png}
\end{figure}
\leavevmode\\
During the implementation of this procedure, a lot of problems of dependency on critical libraries has happened. Starting by cuda, for faster results it is used in all models, but it might not work if the architecture of the graphics card is too old. It is also not compatible with python 3.10 which was the version being worked on at the moment, it had to be switched to an environment with python 3.8 to avoid further issues.
\\
\input{Annex.tex}


\section{Bibliography}
\label{sec-bibliography}

\printbibliography

%\section*{Agraïments}


\end{document}

