\documentclass[10pt,a4paper,twocolumn,twoside]{article}
\usepackage[utf8]{inputenc}
\usepackage[catalan]{babel}
\usepackage{multicol}
\usepackage{graphicx}
\usepackage{fancyhdr}
\usepackage{times}
\usepackage{titlesec}
\usepackage{multirow}
\usepackage{lettrine}
\usepackage[top=2cm, bottom=1.5cm, left=2cm, right=2cm]{geometry}
\usepackage[figurename=Fig.,tablename=TABLE]{caption}

\usepackage[table,xcdraw]{xcolor}
\usepackage{tabularx}
\usepackage{float}

\usepackage{biblatex}
\addbibresource{ref.bib}

\captionsetup[table]{textfont=sc}

\author{\LARGE\sffamily Christian Espinosa Reboredo}
\title{\Huge{\sffamily Recognition of epileptic seizures from EEG data}}
\date{}

\newcommand\blfootnote[1]{%
  \begingroup
  \renewcommand\thefootnote{}\footnote{#1}%
  \addtocounter{footnote}{-1}%
  \endgroup
}

%
%\large\bfseries\sffamily
\titleformat{\section}
{\large\sffamily\scshape\bfseries}
{\textbf{\thesection}}{1em}{}

\begin{document}
\nocite{*}
\fancyhead[LO]{\scriptsize AUTOR: Christian Espinosa Reboredo}
\fancyhead[RO]{\thepage}
\fancyhead[LE]{\thepage}
\fancyhead[RE]{\scriptsize EE/UAB TFG INFORMÀTICA: Recognition of epileptic seizures from EEG data}

\fancyfoot[CO,CE]{}

\fancypagestyle{primerapagina}
{
   \fancyhf{}
   \fancyhead[L]{\scriptsize TFG EN ENGINYERIA INFORMÀTICA, ESCOLA D'ENGINYERIA (EE), UNIVERSITAT AUTÒNOMA DE BARCELONA (UAB)}
   \fancyfoot[C]{\scriptsize Febrer de 2022, Escola d'Enginyeria (UAB)}
}

%\lhead{\thepage}
%\chead{}
%\rhead{\tiny EE/UAB TFG INFORMÀTICA: TÍTOL (ABREUJAT SI ÉS MOLT LLARG)}
%\lhead{ EE/UAB \thepage}
%\lfoot{}
%\cfoot{\tiny{February 2015, Escola d'Enginyeria (UAB)}}
%\rfoot{}
\renewcommand{\headrulewidth}{0pt}
\renewcommand{\footrulewidth}{0pt}
\pagestyle{fancy}

%\thispagestyle{myheadings}
\twocolumn[\begin{@twocolumnfalse}

%\vspace*{-1cm}{\scriptsize TFG EN ENGINYERIA INFORMÀTICA, ESCOLA D'ENGINYERIA (EE), UNIVERSITAT AUTÒNOMA DE BARCELONA (UAB)}

\maketitle

\thispagestyle{primerapagina}
%\twocolumn[\begin{@twocolumnfalse}
%\maketitle
%\begin{abstract}
\begin{center}
\parbox{0.915\textwidth}
{\sffamily
\textbf{Abstract--}An electroencephalogram (EEG) is a test that detects electrical activity of the brain. This paper tries to go a step further to interpret seizures from electroencephalograms using deep learning algorithms. The data used in this paper is a public dataset CHB-MIT\cite{bbddCHBMIT} of recordings of paediatric subjects with intractable seizures. Different methods of data processing are done and documented to make the most of the algorithms used as well as the strategy. The objective is to train an algorithm to classify when the subject is having a seizure and when it is not.
\\
\\
\textbf{Keywords-- }electroencephalogram, deep learning, brain activity, classification, EEG analysis\\
\bigskip
\\
\textbf{Resumen--}Un electroencefalograma (EEG) es una prueba que detecta la actividad eléctrica del cerebro. Este artículo intenta dar un paso más para interpretar los ataques epilépticos a partir de electroencefalogramas utilizando algoritmos de aprendizaje computacional. Los datos utilizados en este documento son de una base de datos pública CHB-MIT\cite{bbddCHBMIT} de EGG de sujetos pediátricos con convulsiones intratables. Se realizan y documentan diferentes métodos de procesamiento de datos para aprovechar al máximo los algoritmos utilizados, así como la estrategia. El objetivo es entrenar un algoritmo para clasificar cuándo el sujeto está teniendo un ataque epiléptico y cuándo no.
\\
\\
\textbf{Palabras clave-- }electroencefalograma, aprendizaje computacional, actividad cerebral, clasificación, análisis EEG\\
}

\bigskip

{\vrule depth 0pt height 0.5pt width 4cm\hspace{7.5pt}%
\raisebox{-3.5pt}{\fontfamily{pzd}\fontencoding{U}\fontseries{m}\fontshape{n}\fontsize{11}{12}\selectfont\char70}%
\hspace{7.5pt}\vrule depth 0pt height 0.5pt width 4cm\relax}

\end{center}

\bigskip
%\end{abstract}
\end{@twocolumnfalse}]

\blfootnote{$\bullet$ E-mail of contact: 1459024@uab.cat}
\blfootnote{$\bullet$ Degree Specialisation taken: Computació }
\blfootnote{$\bullet$ Work supervised by: Aura Hernández Sabaté (Ciencies de la Computació)}
\blfootnote{$\bullet$ Year 2021/22}

\section{Introduction}
\leavevmode\\
\lettrine[lines=3]{A}{n} epileptic seizure is a period of symptoms due to abnormally excessive or synchronous neuronal activity in the brain. This can cause different effects like uncontrolled shaking movements involving much of the body, parts of the body or subtle momentary loss of awareness. In order to understand this issue, it is important to understand how neurons work and interact with each other to conserve what we call consciousness, represented as brain activity and brainwaves.
\\\\
Neural oscillations are rhythmic or repetitive patterns of neural activity in the central nervous system which can be driven by mechanisms within individual neurons or by interactions. Since 1824 neural oscillations have been observed, fifty years later intrinsic oscillatory behaviour was encountered in vertebrate neurons, but the purpose of these is yet to be fully understood.
\\\\
The main objective of this paper is to classify seizures from brain activity by building a deep learning architecture. First of all, it will be needed an inside view on how the brain works to have a hint on how and what is done to extract or intercept information from the neurons to process externally in a computer. This information is available annexed in this paper to have an overview to further understand the subject. This matter is not the main purpose of this paper as the work involves data processing, architecture, model strategies and classification results.
\\\\
In this paper it is detailed the complete process of epileptic seizure detection, from data processing to seizure recognition. This study works as a pipeline of different stages. Therefore, an insight view of each one is done, starting from data processing from a well-known database CHB-MIT of encephalograms collected from 23 subjects with interactable seizures that has been used in previous research. Then followed by the strategy of the architecture used to classify the signals to finally understand the classified results into seizure and not seizure.
\\\\
There have been many other studies about seizure recognition, so in this project another approach is conducted to further study this subject. Before starting the study, an overview of different similar projects has been done. As acknowledgement, the scripts written in this paper have been supervised by the team at Computer Vision Centre (CVC) which are working on a project related to this paper.


\section{Related Work}
A lot of research has been made of the brain, to further understand its capabilities using deep learning algorithms. EEG signals can be used in many ways, such as the ones below. These studies, work on similar objectives of mental workload, seizure classification, attempting to understand and process EEG data, using deep learning techniques to develop frameworks to combine and process these signals.
\\
\subsection{Mental Workload Detection based on EEG Analysis}
\leavevmode\\ 
A study of mental workload\cite{relatedwork0} is done in order to work more efficiently, healthier and to avoid accidents since workload compromises both performance and awareness on the human. The use of EEG signals has a high correlation with specific cognitive and mental states such as workload, proposing a binary neural network to classify EEG features across different mental workloads.
\\\\
Mental workload is defined as “the cognitive and psychological effort to conclude a task”, observing that depending if workload is too heavy or too light it can affect human performance. 
\\\\
There are two main categories to measure workload:
\\
\begin{itemize}
\item \textbf{Subjective measures:} Being the most used to asses mental workload, the NASA Task Load Index (TLX)\cite{tlx} a prominent way to gain insight on perceived workload from the subject based on a weighted average of six sub-variables: mental demand, physical demand, temporal demand, performance, effort and frustration. 
\item \textbf{Physiological measures:} They provide a more reliable data by measuring physiological dynamic changes which cannot be controlled consciously, so that is why it is more reliable. Some examples of these readings are electrocardiogram (ECG), electromyograph, electroencephalogram (EEG)... The combination of inputs reports better accuracy than the analysis if each one independently.
\end{itemize}

\leavevmode\\
The approach of the mental workload detection based on EEG analysis study is to investigate the ability of one dimensional convolutional neural network (1D-CNN)\cite{CNN} models to recognise two types of mental load from EEG signals and to generalise the model to a population that has not been take into account in the training set. N-back test (memory demanding games requiring the resolution of simple arithmetic operations adjusting workload) is used to induce low and medium workload and to classify a simple neural network (NN) trained using only the power spectrum of theta waves.
\\\\
The method to obtain data started by having the sixteen subjects to watch a 10-minute relaxing video. Then, perform the N-back-test low, medium and high difficulty tests. Finally, subjects fill in a TLX questionnaire for subjective perception of the test difficulty and workload.
\\\\
To obtain the data, EEG recordings were done using an EMOTIV EPOC+ headset which has 14 electrodes placed according to the 10/20 system, which provides raw data and power spectrum for the main brain rhythms (theta, alpha, beta low, beta high), at 128 Hz and 8 Hz, respectively. 
\\\\
The study concluded with the following results, reported 95\% confidence interval for each class computed for all subjects and recall above 90\%. It is suggested to use longer windows to capture EEG non stationary nature.
\\


\subsection{EEG Signal Dimensionality Reduction And Classification Using Tensor Decomposition And Deep Convolutional Neural Networks}
\leavevmode\\
In this study, EEG Signal Dimensionality Reduction and Classification Using Tensor Decomposition and Deep Convolutional Neural Networks\cite{relatedwork1} has the same objective of seizure classification. It is the only project I have been able to find using the same dataset CHB-MIT. The study uses convolutional networks (CNN) but with different architectures.
\\\\
Using convolutional neural networks (CNNs), still suffer from high dimensionality of the training data. The proposed tensor decomposition-based dimensionality reduction algorithm transforms a large set of slices of the input tensor to a concise set of slices which are called super-slices which handles the artifacts and redundancies of the EEG data and also reduces the dimension of the CNNs training inputs. This proposed framework is tested on HCB-MIT data and as results show, the approach outperforms other previous studies.
\\


\subsection{Epileptic Seizures Detection Using Deep Learning Techniques: A Review}
\leavevmode\\
The development of deep learning algorithms in many areas of medicine, such as in the diagnosis of epileptic seizures, has made significant advances.
\\\\
In this study Epileptic Seizures Detection Using Deep Learning Techniques\cite{relatedwork2}, a comprehensive overview of works focused on automated epileptic seizure detection using deep learning techniques and neuroimaging modalities is presented. Various methods proposed to diagnose epileptic seizures automatically using EEG and magnetic resonance imaging (MRI) modalities are described. In addition, rehabilitation systems developed for epileptic seizures using deep learning have been analysed, and a summary is provided. The rehabilitation tools include cloud computing techniques and hardware required for implementation of deep learning algorithms. Challenges, advantages and limitations in employing deep learning-based techniques for epileptic seizures diagnosis are presented as well as the most promising models and future works on automated epileptic seizure detection.
\\


\subsection{Recognition of mental workload of pilots in the cockpit using EEG signals}
\leavevmode\\
The commercial flightdeck is a naturally multi-tasking work environment and so automatic characterization of pilot’s workload becomes essential. Electroencephalogram (EEG) signals have shown a high correlation to specific cognitive and mental states like workload. However, there is not enough evidence in the literature to validate how well models generalize in case of new subjects performing tasks of a workload similar to the ones included during model’s training.
\\\\
In this paper Recognition of mental workload of pilots in the cockpit using EEG signals\cite{relatedwork3} it is proposed a convolutional neural network to classify EEG features across different mental workloads in a continuous performance task test that measures a portion of working memory and working memory capacity. The goal of this paper is to characterize the mental workload of flying pilots in the cockpit from the analysis of EEG signals. The model proposed is valid at a general population level and it is able to transfer task learning to a pilot mental workload recognition in a simulated operational environment.
\\
\section{Objectives}
\label{sec-objectives}
The main goal in this study is to detect seizures from the dataset CHB-MIT. To fulfil the objective an architecture has been created working within a pipeline of events. Starting by finding the best data analysis and processing method before feeding it to the deep learning algorithms. There are many sub-objectives to be completed to obtain good results from this architecture and also, it’s modular for further expansion and studies. The main strategy of the scripts execute in sequence (Fig. 1)
\\

Within the objective of data classification, another objective to find the most precise architecture is defined to obtain results with the most accuracy possible. In order to fulfil these objectives a set of sub objectives have been defined. The following points summarize the set of subobjectives:
\\
\begin{enumerate}

\item Raw data must be readable, as the data base CHB-MIT is in European Data Format (EDF), a standard file format designed for exchange and storage of medical time series, all files in the dataset are in edf format. A script has been programmed to save edf files into parquet format. Parquet is needed to be able to handle data more easily, as well as improving compatibility with other related work scripts from the CVC.
\item Setting different functions to filter data, making sure data fits certain constraints to obtain better results when training the model. This makes the script modular to obtain data from the dataset CHB-MIT.
\item Label raw data, to have a ground truth from the provided summary files in the dataset.  This part is essential to understand if the model works as expected.
\item Functions create constraints on the dimensionality of the data to fit the input of the model as well as filtering data from the dataset. This functions filter in bandwidth of the data, and excluding files from the dataset where data is not well recorded.
\item Each model needs to be configured to accept the dimensionality of the data fed to it. All data must be in two files of numpy arrays and all input must have the same dimensions. This will make sure all files are read the same way, and the model will train with uniform data.
\item Work different models to choose what models give better answers form input data. The CVC provides different models adapted to related work. These models have different inputs and are initialized in different ways. All models must be initialized and executed the same way, to make it easier to train and test more than one model, to obtain different answers.
\item After all the models results, an overview is done to understand the results and conclude the best way to treat this database, for further investigation.
    
\end{enumerate}
\begin{figure}[H]
    \caption{Scheme of data processing }
    \centering
    \includegraphics[width=0.5\textwidth]{img/pipeline.png}
\end{figure}
\leavevmode\\
\input{3methods.tex}
\section{Execution}
\leavevmode\\
The execution of the model has been made with all the previous procedures regarding data processing in this paper. Data from subject 1 to 10 has been used to train the model and data from 11 to 16 has been used to test the accuracy of the predictions of the model. For every file 50 epochs have been done to ensure a good understanding from the model of the data and a (3,1) kernel size.
\\\\
At first the strategy of training and testing the model was done within the files of the subjects. For each file it was split in two by a percentage variable and afterword’s it was trained and tested. This strategy was abandoned because it wasn’t as efficient as if many files where trained. Because there are few seizures in the hole database it makes it difficult to train a model to understand the existence of a seizure. Also, it was dependent on the position of data in the file, considering there are few or in most cases only one seizure, if the file was split one half would have a seizure and the other would not have one. This is a big problem if training has no seizure and the testing data has it, it will never be able to learn what a seizure is, and if its opposite, it might learn what a seizure is but it would never be possible to test if it learned correctly to classify.
\leavevmode\\
\begin{figure}[h!]
  \caption{Data from subject 12 file 8 seen scattered through the hole recording}
  \centering
  \includegraphics[width=0.5\textwidth]{img/12_8-elecFP1-F7.png}
\end{figure}

\begin{figure}[h!]
  \caption{Data from subject 1 file 3 with only one seizure. This file wouldn't be eligible to split in train and test }
  \centering
  \includegraphics[width=0.5\textwidth]{img/1_3-elecFP1-F7.png}
\end{figure}
\leavevmode\\
In the end the strategy of first training the model with some files and then with others do the testing, was the best way to go considering training each file took around 10 min depending on the computer it was executed with. There was a big difference between executing the model with and without cuda, the first model was executed without cuda because of the hardware architecture and it spent around 30 hours training the model, and with cuda half the time. It’s important to note using a desktop computer to train the model, considering it will get hot finishing the last epochs of every file.
\\\\
It’s also very resource consuming, considering it loads all the file to RAM reading and working with it from there. It could be a problem to many computers, because the one’s with 8Gb won’t work, only with 16Gb of memory worked for me. In most cases, the edf files contain exactly one hour of digitized EEG signals, although those belonging to case chb10 are two hours long, and those belonging to cases chb04, chb06, chb07, chb09, and chb23 are four hours long; occasionally, files in which seizures are recorded are shorter, so the idea of concatenating files of a subject to have one file per subject it’s also excluded. 
\leavevmode\\


\section{Conclusion}
\subsection{Results}
\leavevmode\\
After training the two models with files from subjects 1 to 10, testing only took 15 to 30min to execute with subjects from 11 to 16. The results given by the function classification\_report of the library metrics from sklearn were saved in parquets and then loaded with “05\_ReadParquet.py” script, which reads every parquet from the folder results of every tested subject. 
\\\\
At first results looked promising viewing a staggering 95\% to 100\% accuracy in classification from the model. But upon further inspection the results are bad, because I realized the accuracy given by the report was because it predicted all data to be “no seizure”. So, the other 5% it’s the Seizure in the data that the model is classifying it as no-seizure. Because data is mainly one-sided to no-seizure the result 95% is because it was only classifying this class, so in the end it’s returning the percentage of the  amount of data of each class.
\\\\
Looking up the results and the model I realized the data in the input of the model is not exactly as expected. There are inconsistencies in the results, for example as shown in the next two tables the two tables should have the same dimensions but here it’s not the case:
\leavevmode\\

\begin{table}[H]
    \caption{Results from subject chb14 file 6}
    \begin{tabularx}{\columnwidth}{ @{\extracolsep{\fill}} |c|c|c|c|}
        \hline
                        & \textbf{0} & \textbf{1}                 & \textbf{accuracy} \\ \hline
        \textbf{precision} & 0.983      & {\color[HTML]{FE0000} 0.0} & 0.983             \\ \hline
        \textbf{recall}    & 1.000      & {\color[HTML]{FE0000} 0.0} & 0.994             \\ \hline
        \textbf{f1-score}  & 0.997      & {\color[HTML]{FE0000} 0.0} & 0.994             \\ \hline
        \textbf{support}   & 358.00     & 2.0                        & 0.994             \\ \hline
    \end{tabularx}
\end{table}

\begin{table}[H]
    \centering
    \caption{Results from subject chb14 file 11}
    \begin{tabular}{|l|l|l|}
        \hline
                        & \textbf{0} & \textbf{accuracy} \\ \hline
        \textbf{precision} & 1.0        & 1.0               \\ \hline
        \textbf{f1-score}  & 1.0        & 1.0               \\ \hline
        \textbf{support}   & 360.0      & 1.0               \\ \hline
    \end{tabular}
\end{table}
\leavevmode\\

\subsection{Future Work}
\leavevmode\\
For future work it would be strictly necessary first of all, to change the chunker function defining the windows of all the files. So assuming the model works as it should, getting the right processed data would be enough to have a consistent result. If the result of the classification is not good enough, I would consider adding a weighted cross entropy loss to avoid so much one sided data, for the model to learn more uniformly.
\\\\
When training and testing the model, it would be a good idea to only do these two procedures. A good way to avoid extending training and testing execution time would be to normalize data before any model execution. This way all files, would be normalized considering all files and not just only one, like it’s done in the execution script. As previously mentioned, the scalers are obtained by only adding one file. An average of scalers should be considered to keep in touch with the type of signals of every file, especially normalizing data between subjects, because seizures could change between different people and have a different impact on the model.
\\\\
With better results the code could give more accurate results so the difference between models could be further proven, to understand the best strategy to obtain process data. Not only two models but all the models offered by the CVC (CNN\_ConcatInput, CNN\_ProjOut\_Conv, CNN\_ProjOut\_Concat, CNN\_ProjOut\_AvgW, CNN\_ProjChannel, CNN\_ProjChannel\_v2, Seq\_C1D, Seq\_C1D\_Ensemble).
\\\\
Regarding biological characteristics, other models that consider data sequences could be also utilised such as dominant sequence transduction models, like states Attention is All you Need paper. Creating a model and maybe consider classifying data in three classes, seizure, no seizure and pre-seizure. This would enable people to predict seizures and further understand the reason of them if these are linked to a sequence or pattern. Because this type of algorithm is used to understand sequences, such as natural language processing, if the existence of seizures are linked to a history of patterns that could be recorded in encephalograms and processed to predict when it’s likely to be another seizure. Models could be trained and future devises could be programmed and created to at least advice or warn the subject a seizure is about to appear. So other problems could be avoided because the subjects could brace themselves, and take precautions such as seating down, closing the mouth, or simply get in a position more secure before the attack. 
\\\\
Data processing with other ranges of frequencies would be another issue of research as well, because in this paper only theta frequencies are considered and all other excluded, but there’s also delta, alpha, beta, gamma to research with, and it would be interesting to find out if a model could learn better from other bandwidths, and classify which one’s are better for classifying seizures.
\\\\
The position of the electrodes has been taken care of in the processing script, in a way where if there is an interruption and electrodes are changed, then the file was automatically excluded. But this could be changed to match the data of the previous positions of the electrodes and the positions of the one’s after the interruption. This way more data could be used to train the models as well as maybe considering different Brodmann’s areas (explained in the annex), to give more importance to data coming from certain areas in the brain which could cause or give more information about the reason, detonation or existence of seizures.
\\\\
As this paper the model was trained with different subjects because there’s not enough data to train with, it should be considered if different people have the same “type of seizure” affecting in the learning of the model. If enough data from only one subject was enough to train the model, it should be considered if this model could also predict seizures in other subjects with the same problem. The seizure itself might change between subjects, so maybe it’s alright to train the model for now with different subjects to consider many possibilities. 
\\\\
Finally other datasets should be researched as well to have a big overview of the difference in data between datasets regarding the seizure issue. It’s hard to come by with any, but it could help a lot with the learning of models to find out how seizures come to be. Also, other scripts should be created to process data for the model to input the same way, with the same characteristics, as it’s done with this dataset. For sure, this would be much more time consuming but with promising results.
\\

\section{Acknowledgments}
\leavevmode\\
In the production of this TFG there has been a lot of people involved helping me to make this project as it is right now. When I first started working with Aura Hernandez and Debora Gil, they provided me with the code of the models to change and adapt to the dataset CHB-MIT I have been working with in this TFG. Because this team is working on the Mental Workload Detection, they provided me with all sorts of scripts in models and data processing. I was able to make an idea of what I wanted, but because of the big difference on the datasets, data processing scripts had to be redone completely as well as modifying the models to fit and work with the new database.
\\\\
Once I was working with this TFG, mid-way I started to have problems with my lack of knowledge on how torch works. I desperately needed an in-depth insight on how the models were created and the reason of the internal structure. With Jose Elias we have been doing all kind of reunions to further understand these issues and also consider other strategies on how to process data more efficiently. Being able to work with someone who already has experience in this field of artificial intelligence has given a boost with the development of this project my own personal knowledge.
\\\\
I would like to consider the effectiveness in response from Jordi Pons when there’s a problem or simply any doubt. It’s someone to surely rely on, he’s always around and willing to help, this attitude is well known and appreciated by all.
\\\\
Last but not list, special thanks to my teacher Aura Hernandez helping me review, schedule and manage this project as important as it is. Training models, providing information and guidance on any topic, followed by a big interest in the subject and charisma, makes difficult times easier to overcome.
\\\\
Much appreciated everyone who has showed interest in this project. I wouldn’t have done it without the help of all professionals involved in this project, thank you all.
\\


\section{Bibliography}
\label{sec-bibliography}

\printbibliography

\section{Annex}
\label{sec-annex}

\subsection{Neuron and neural activity}
\label{subsec-neuron}
To further understand how brain activity works we first need to study a single neuron and its purpose. A neuron is an electrically excitable cell that has the function to communicate with other cells. It does it by nearly touching other cells called synapsis. It transmits the message through its axon and delivers the message by synapsis\cite{Synapse} to another cell. Neurons are typically classified into types based on their function:
\\
\begin{itemize}
  \item \textbf{Sensory neurons\cite{Sensoryneuron}:} Which respond to stimuli of the sensory organs and send the signals to the spinal cord or brain.
  \item \textbf{Motor neurons\cite{Motorneuron}:} Its axons originate in the brain and spinal cord and innervate the muscles to produce muscle movements.
  \item \textbf{Projection fiber\cite{Projectionfiber}:} These types of neurons are found in the central nervous system and only establish synapses with other neurons, consisting of efferent and afferent fibers uniting the cortex with the lower parts of the brain and with the spinal cord.
  \item \textbf{Interneuron\cite{Interneuron}:} Is a neuron of the central nervous system, usually small and with a short axon. It interconnects with other neurons, but never with sensory receptors or muscle fibres, allowing it to perform more complex functions.
\end{itemize}
\leavevmode\\
Neurons transmit electrical waves originating from a transient change of permeability in the plasma membrane. Their propagation is due to the existence of a potential difference that arises from different concentrations of ions on either side of the membrane, as described by the Nernst potential, between the inner and outer part of the cell (typically -70 mV). For the transmission of nervous impulses to other neurons, these do it by synapse, being a structure to pass electrical or chemical signals to another neuron or effector cell, there are two types of synapses:\cite{Synapse}:
\\
\begin{itemize}
  \item	\textbf{Chemical synapse:} Electrical activity in the presynaptic neuron is converted into the release of a neurotransmitter that binds to the receptors located in the plasma membrane of the postsynaptic cell.
  \item	\textbf{Electrical synapse:} Transmission between the first neuron and the second is not by the secretion of a neurotransmitter, but by the passage of ions from one cell to another through gap junctions. Small channels formed by the coupling of protein complexes, based on connexins, in closely adherent cells.
\end{itemize}

\leavevmode\\
These electrochemical processes when large numbers of neurons show synchronized activity, electric fields that they generate can be large enough to be detected outside the skull, and so using electroencephalography (EEG) or magnetoencephalography (MEG) brain activity can be recorded.
\\

\subsection{Structure}
\label{subsec-structure}
\leavevmode\\
Now that we know where brain activity originates from, we can further study how the brain structures. There are many parts in the brain, but for now we are going to focus on the cerebrum because it initiates and coordinates movement, regulates temperatures, speech, judgement, reasoning, problem-solving, emotions, learning…
\\\\
The cerebrum\cite{Cerebrum}, it is the largest part of the brain, it is divided by the medial longitudinal fissure in two hemispheres. Each of these hemispheres, has an outer layer of grey matter the cerebral cortex, and an inner layer of white matter. The fact that these are separated gives the opportunity for lateralisation of brain functions, which is the tendency of neurological functions to specialise in one hemisphere or the other, but even though the cerebrum is separated, these are connected by the corpus callosum.
\\\\
The cortex is mapped into fifty different functional areas known as Brodmann’s areas\cite{Brodmannarea}, defined by its cytoarchitecture (cellular composition), or histological structure and organization of cells. One scheme widely used (from Korbinian Brodmann) splits the cortex into 52 different numbered areas of different cellular structure and different functions.
\\

\begin{figure}[H]
  \caption{Brodmann areas by colors}
  \centering
  \includegraphics[width=4cm]{img/Brodmann_areas.png}
\end{figure}
\leavevmode\\
Having clarified this brain structure, obtaining data with electrodes from brain activity, and the positioning of these is something to keep in touch with depending on what it is being studied.
\\

%\section*{Agraïments}


\end{document}

