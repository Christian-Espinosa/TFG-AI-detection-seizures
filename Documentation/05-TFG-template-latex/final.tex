\section{Execution}
\leavevmode\\
The execution of the model has been made with all the previous procedures regarding data processing in this paper. Data from subject 1 to 10 has been used to train the model and data from 11 to 16 has been used to test the accuracy of the predictions of the model. For every file 50 epochs have been done to ensure a good understanding from the model of the data and a (3,1) kernel size.
\\\\
At first the strategy of training and testing the model was done within the files of the subjects. For each file it was split in two by a percentage variable and afterword’s it was trained and tested. This strategy was abandoned because it wasn’t as efficient as if many files where trained. Because there are few seizures in the hole database it makes it difficult to train a model to understand the existence of a seizure. Also, it was dependent on the position of data in the file, considering there are few or in most cases only one seizure, if the file was split one half would have a seizure and the other would not have one. This is a big problem if training has no seizure and the testing data has it, it will never be able to learn what a seizure is, and if its opposite, it might learn what a seizure is but it would never be possible to test if it learned correctly to classify.
\\\\
In the end the strategy of first training the model with some files and then with others do the testing, was the best way to go considering training each file took around 10 min depending on the computer it was executed with. There was a big difference between executing the model with and without cuda, the first model was executed without cuda because of the hardware architecture and it spent around 30 hours training the model, and with cuda half the time.
\\\\
It’s important to note using a desktop computer to train the model, considering it will get hot finishing the last epochs of every file. It’s also very resource consuming, considering it loads all the file to the RAM and reads and works with it from there. It could be a problem to many computers, some with 8Gb won’t work, only with 16Gb of memory worked for me. In most cases, the .edf files contain exactly one hour of digitized EEG signals, although those belonging to case chb10 are two hours long, and those belonging to cases chb04, chb06, chb07, chb09, and chb23 are four hours long; occasionally, files in which seizures are recorded are shorter, so the idea of concatenating files of a subject to have one file per subject it’s also excluded. 
\\


\section{Conclusion}
\subsection{Results}
\leavevmode\\
After training the two models with files from subjects 1 to 10, testing only took 15 to 30min to execute with subjects from 11 to 16. The results given by the function classification\_report of the library metrics from sklearn were saved in parquets and then loaded with “05\_ReadParquet.py” script, which reads every parquet from the folder results of every tested subject. 
\\\\
At first results looked promising viewing a staggering 95\% to 100\% accuracy in classification from the model. But upon further inspection the results are bad, because I realized the accuracy given by the report was because it predicted all data to be “no seizure”. So, the other 5% it’s the Seizure in the data that the model is classifying it as no-seizure. Because data is mainly one-sided to no-seizure the result 95% is because it was only classifying this class, so in the end it’s returning the percentage of the  amount of data of each class.
\\\\
Looking up the results and the model I realized the data in the input of the model is not exactly as expected. There are inconsistencies in the results, for example as shown in the next two tables the two tables should have the same dimensions but here it’s not the case:
\\

\begin{table}[]
    \caption{Results from subject chb14 file 6}
    \begin{tabularx}{\columnwidth}{ @{\extracolsep{\fill}} |c|c|c|c|}
        \hline
                        & \textbf{0} & \textbf{1}                 & \textbf{accuracy} \\ \hline
        \textbf{precision} & 0.983      & {\color[HTML]{FE0000} 0.0} & 0.983             \\ \hline
        \textbf{recall}    & 1.000      & {\color[HTML]{FE0000} 0.0} & 0.994             \\ \hline
        \textbf{f1-score}  & 0.997      & {\color[HTML]{FE0000} 0.0} & 0.994             \\ \hline
        \textbf{support}   & 358.00     & 2.0                        & 0.994             \\ \hline
    \end{tabularx}
\end{table}

\begin{table}[]
    \centering
    \caption{Results from subject chb14 file 11}
    \begin{tabular}{|l|l|l|}
        \hline
                        & \textbf{0} & \textbf{accuracy} \\ \hline
        \textbf{precision} & 1.0        & 1.0               \\ \hline
        \textbf{f1-score}  & 1.0        & 1.0               \\ \hline
        \textbf{support}   & 360.0      & 1.0               \\ \hline
    \end{tabular}
\end{table}


\subsection{Future Work}
\leavevmode\\
For future work it would be strictly necessary first of all, to change the chunker function defining the windows of all the files. So assuming the model works as it should, getting the right processed data would be enough to have a consistent result. If the result of the classification is not good enough, I would consider adding a weighted cross entropy loss to avoid so much one sided data, for the model to learn more uniformly.
\\\\
With better results the code could give more accurate results so the difference between models could be further proven, to understand the best strategy to obtain process data. Not only two models but all the models offered by the CVC (CNN\_ConcatInput, CNN\_ProjOut\_Conv, CNN\_ProjOut\_Concat, CNN\_ProjOut\_AvgW, CNN\_ProjChannel, CNN\_ProjChannel\_v2, Seq\_C1D, Seq\_C1D\_Ensemble).
\\\\
Regarding biological characteristics, other models that consider data sequences could be also utilised such as dominant sequence transduction models, like states Attention is All you Need paper to create a model and maybe consider classifying data in three classes, seizure, no seizure and pre-seizure. This would enable people to predict seizures and further understand the reason of them if these are linked to a sequence or pattern.
\\\\
Data processing with other ranges of frequencies would be another issue of research as well, because in this paper only theta frequencies are considered and all other excluded, but there’s also delta, alpha, beta, gamma to research with, and it would be interesting to find out if a model could learn better from other bandwidths, and classify which one’s are better for classifying seizures.
\\\\
The position of the electrodes has been taken care of in the processing script, in a way where if there is an interruption and electrodes are changed, then the file was automatically excluded. But this could be changed to match the data of the previous positions of the electrodes and the positions of the one’s after the interruption. This way more data could be used to train the models as well as maybe considering different Brodmann’s areas (explained in the annex), to give more importance to data coming from certain areas in the brain which could cause or give more information about the reason, detonation or existence of seizures.
\\\\
As this paper the model was trained with different subjects because there’s not enough data to train with, it should be considered if different people have the same “type of seizure” affecting in the learning of the model. If enough data from only one subject was enough to train the model, it should be considered if this model could also predict seizures in other subjects with the same problem. The seizure itself might change between subjects, so maybe it’s alright to train the model for now with different subjects to consider many possibilities. 
\\\\
Finally other datasets should be researched as well to have a big overview of the difference in data between datasets regarding the seizure issue. It’s hard to come by with any, but it could help a lot in the learning of models to find out how seizures come to be. Also, other scripts should be created to process data for the model to input the same way, with the same characteristics, as it’s done with this dataset. For sure, this would be much more time consuming.
\\

\section{Acknowledgments}
\leavevmode\\
In the production of this TFG there has been a lot of people involved helping me to make this project as it is right now. When I first started working with it Aura Hernandez and Debora Gil provided me with the code of the models to change and adapt to the dataset I have been working with in this TFG. Because this team is working on the Mental Workload Detection the they provided me with al sorts of scripts in models and data processing. I was able to make an idea of what I wanted, but because of the big difference on de datasets, the data processing scripts had to be redone completely.
\\\\
Once I was working with this TFG, mid-way I started to have problems in my lack of knowledge on how torch works and I desperately needed an in-depth insight on how the models were created and the reason of the internal structure. With Elias we have been doing all kind of reunions to further understand these issues and also considering other strategies on how to process data more efficiently.
\\\\
I would like to consider the effectiveness in response from Jordi Pons when there’s a problem or simply any doubt. It’s someone to surely rely one, he’s always around and willing to help, this attitude is well known and appreciated by all.
\\\\
Last but not list special thanks to my teacher Aura Hernandez helping me review, schedule and manage this project as important as it is. Much appreciated everyone who has showed interest in this project. I wouldn’t have done it, with any doubt, without the help of people so professional I have been working with, thank you all.
\\