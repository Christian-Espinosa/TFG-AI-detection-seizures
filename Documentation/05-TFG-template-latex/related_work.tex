\section{Related Work}
\label{sec-related-work}
A lot of research has been made of the brain to further understand it’s capabilities using deep learning algorithms. Below are different projects attempting to interpret and process EEG data in order to define a baseline of what has been done so far. 
\\
\subsection{Mental Workload Detection based on EEG Analysis}
\label{subsec-work1}
A study of mental workload is done in order to work more efficiently, healthier and to avoid accidents since workload compromises both performance and awareness. The use of EEG signals has a high correlation with specific cognitive and mental states such as workload, proposing a binary neural network to classify EEG features across different mental workloads.
\\
Mental workload is defined as “the cognitive and psychological effort to conclude a task”, observing that depending if workload is too heavy or too light this can affect human performance. Also, since workload involves neuro-physiologic, and perceptual processes it is affected by individual capabilities, motivation to preform, physical and emotional state which it’s multifaceted nature of workload prevents it of studying directly but it is feasible to infer this from a number of quantifiable variables.
\\
There are two main categories to measure workload:
\\
\begin{itemize}
\item Subjective measures: Being the most used to asses mental workload, the NASA Task Load Index (TLX) a prominent way to gain insight on perceived workload from the subject based on a weighted average of six sub-variables: mental demand, physical demand, temporal demand, performance, effort and frustration. Widely used in aviation to assess mental workload of the pilot but it is highly subjective.
\item Physiological measures: Providing a more reliable data by measuring physiological dynamic changes which cannot be controlled consciously, so that is why it’s more reliable. Readings such as electrocardiogram (ECG), electromyograph, electroencephalogram (EEG), photoplethysmography, respiration rate sensors, electro-dermal activity (EDA), oxygen density in the blood in the brain, and eye movement trackers... The combination of inputs reports better accuracy than the analysis if each one independently.
\end{itemize}
\leavevmode\\
The approach is to investigate the ability of 1D-CNN models to recognise two types of mental load from EEG signals and to generalise the model to a population not seen in the training set. They use N-back test (memory demanding games requiring the resolution of simple arithmetic operations adjusting workload) to induce low and medium workload and to classify a simple neural network (NN) it’s trained using only the power spectrum of theta waves. Proposing a personalized model for each individual and a generalist one, preform the models in a dataset of 16 subjects showing outstanding results in a leave-one-out subject test and so generalizing to new unseen subjects.
\\
The method to obtain data all 16 subjects had to first watch a 10-minute relaxing video and afterwards do the N-back-test low, medium and high difficulty. Finally subjects ask a TLX questionnaire for subjective perception of the test difficulty and workload.
\\
To get the data itself EEG recordings were done using EMOTIV EPOC+ headset which has 14 electrodes placed according to the 10/20 system which provide raw data and power spectrum for the main brain rhythms (theta, alpha, beta low, beta high, and), at 128 Hz and 8 Hz, respectively. Data had to be filtered from noise with Inter Quartile Range (IQR) strategy to detect outlier values associated to muscular movement wave peaks. In this work, power spectrum used was from theta wave (4-8Hz) sampled at 8Hz feed into the models in 5 second windows of every electrode (14 EEG sensors). Variables were normalized to have 0 mean and
sigma=1 using the mean and standard deviation of the training set. 
\\
Networks have a hidden layer of 128 neurons with ReLU as the activation function. Before the classification layer a dropout layer was added to avoid overfitting and all models have been trained using weighted cross-entropy loss, to compensate the different lengths of base line and workload phases which introduces unbalance in data samples.
\\
In conclusion its reported 95\% confidence interval for each class computed for all subjects and recall above 90\%. It is suggested to use longer windows to capture EEG non stationary nature.
\\
%\subsection{EEG SIGNAL DIMENSIONALITY REDUCTION AND CLASSIFICATION USING TENSOR DECOMPOSITION AND DEEP CONVOLUTIONAL NEURAL NETWORKS}
%\label{subsec-work2}